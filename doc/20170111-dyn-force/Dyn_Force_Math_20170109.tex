\documentclass[10pt,titlepage]{article}
\usepackage[utf8]{inputenc}
\usepackage[T1]{fontenc}
\usepackage{lmodern}
\usepackage[english]{babel}

\usepackage[a4paper]{geometry}
\usepackage{amsmath}
\usepackage{amsfonts}
\usepackage{amssymb}
\usepackage{amstext}
\usepackage{mathtools}

\begin{document}

\author{Ottmar K. Kechel}
\date{Datum}


{\large Force on Charges}\\\\
Author: Ottmar Kechel        
\\
\\
Version 10 Jan 2017\\\\
Derivation of the new formula 
\\\\
The force on charges is given by few equations.
The formulas are taken from 
\\John David Jackson [JDJ]; Classical Electrodynamics; Third Edition.
\\The F numbers refer to the Formula No. (F n.m) in that book.
\\

1. The Lorentz Force (Introduction F 1.3)


\begin{equation}
\textbf{F} = q(\textbf{E}+\textbf{v }\textbf{x}\textbf{ B})
\end{equation}\\

2. The Coulomb Formula gives the static force between two charges $q_1$ and $q_2$ 
\\

\begin{equation}
\textbf{F} = \frac{1}{(4\pi \epsilon)} (q_1*q_2)\frac{\textbf{r}}{\vert\textbf{r}\vert^3}
\end{equation}\\

as  $c = ( \mu_0 * \mu_0)^{-1/2}$        according JDJ page 3 and $\mu_0=10^{-7}$ in SI units
\\
the expression $\frac{1}{4\pi\epsilon}$ can and will be replaced in the following by $c^2/10^7$\\
\\
so the Coulomb law has the following expression (Chapter 1  F 1.3)
\begin{equation}
\textbf{E} = \frac{ q_1c^2}{10^7}\frac{\textbf{x-x1}}{\vert\textbf{x-x1}\vert^3}
\end{equation}\\

3. The BIOT and Savart Formula (Chapter 5  F 5.5)
\\(By the way the same formula appears in chapter 11.10 in the Gauss unit system describing the B field of a passing charge.)\\
\\
\begin{equation}
\textbf{B} = kq \frac{\textbf{v} \times \textbf{x}}{\vert\textbf{x}\vert^3}
\end{equation}
\\
The constant factor k is in SI  $10^{-7}$ with the units $[N/A^2]$ =  $[kg*m/C^2]$
\\\\
Here a warning is included in the book
"... But this expression is time dependent and furthermore is valid only for charges whose velocities are small compared to that of light and whose accelerations can be neglected .  ..."

a) In our New Theory the speed of light is no limit for particles, so that this limitation is not valid in the New Theory. 

b) For the acceleration the present interpretation is also not sufficient, as it is known, that the rotation of electrons in atoms around the kernel cannot be correctly interpreted by the present theory. The electrons are accelerated continuously and should emit electromagnetic waves. They do not ! So the present very simple rule, that accelerated charges create electromagnetic waves needs also a refinement. 
(We are confident that we find a sound explanation in the New Theory. This explanation is under work.)\\
\\
With the replacement as before the Biot Savart formula becomes
\begin{equation}
\textbf{B} = \frac{q}{10^7} \frac{\textbf{v} \times \textbf{x}}{\vert\textbf{x}\vert^3}
\end{equation}\\
\\
\\
The interpretation of (5) is:\\
a charge q generates a \textbf{B} field at the observation point, when it has the velocity \textbf{v} and its location is described by the distance vector \textbf{x}.
\\
\\
Now one can combine the Biot and Savart and Coulomb formula with the Lorentz force.

\begin{equation}
\textbf{F} = q(\textbf{E}+\textbf{v }\textbf{x}(\frac{1}{10^7} \frac{\textbf{v} \times \textbf{x}}{\vert\textbf{x}\vert^3}))
\end{equation}\\

For better reading \textbf{x} will be replaced by  \textbf{r} and the term for the static force included

\begin{equation}
\textbf{F} = \frac{{q_1}{q_2}}{{10^7}}( \frac{c^2\textbf{r}}{\vert\textbf{r}\vert^3}+\textbf{v }\textbf{x}( \frac{\textbf{v} \times  \textbf{r}}{\vert\textbf{r}\vert^3}))
\end{equation}\\

This is equal to

\begin{equation}
\textbf{F} = \frac{q_1q_2}{10^7\vert\textbf{r}\vert^3}  (c^2\textbf{r}+\textbf{v }\textbf{x }( {\textbf{v} \times  \textbf{r}}))
\end{equation}\\
or
\begin{equation}
\textbf{F} = \frac{q_1q_2c^2}{10^7\vert\textbf{r}\vert^3}  (\textbf{r}+\frac{1}{c^2}(\textbf{v }\times( {\textbf{v} \times  \textbf{r}}))
\end{equation}\\
or
\begin{equation}
\textbf{F} = \frac{q_1q_2c^2}{10^7\vert\textbf{r}\vert^3}  (\textbf{r}+(\frac{\textbf{v}}{c}\times( {\frac{\textbf{v}}{c}    \times {\textbf{r}}}))
\end{equation}
\\


Now one can transform the two vector products according the mathematical identity



\begin{equation}
\textbf{a }\times (\textbf{ b } \times  \textbf{c })=\textbf{ b }(\textbf{a }\textbf{c }) - \textbf{c }(\textbf{a }\textbf{ b })
\end{equation}\\
and gets
\begin{equation}
\textbf{F} = \frac{q_1q_2}{10^7\vert\textbf{r}\vert^3}  (c^2\textbf{r}  +   \textbf{ v }(\textbf{v }\textbf{r }) - \textbf{r }(\textbf{v }\textbf{ v }))
\end{equation}\\
\\
\\
From the formulas (7) to (10) follows
\\a) the force is always proportional to the product of the charges (for static and dynamic forces)
\\b) the force decreases always with the square of the distance )for static and dynamic forces)
\\c) there are always 2 components included, the first expression in the brackets corresponds to the static force between the charges, the second expression represents the dynamic force, what usually is called the magnetic influence.  
\\d) the first cross product generates a force that is always perpendicular to the momentary speed
\\
\\
From the formula (12) follows
\\e) That the dynamic force is zero whenever the velocity \textbf{v } and radius vector \textbf{r } are parallel (this includes on the same line) to each other.
\\
\\
With that it is shown, that the forces between two charges moving relative to each other can be calculated with formula (12). The result is identical to a calculation where at first the B field is determined then  the resulting force calculated.
But this expression does not include anymore a B-field expression. The calculation of a B field is obsolete.
It is replaced by the dynamical force between two charges.
Also the use of the three finger rule to determin the resulting force direction is no longer needed. The resulting dynamical force is always in the plane given by the velocity vector and the distance vector.
This means that a B field is not existing, it is only a helpful description for many calculations, especially, when regarding the movement of a low mass single charged particle influenced by many other charges with higher mass, so that the backward interaction can be neglected.
\\
\\
Now lets look at some more details.
As there is no absolute velocity, and no limit in the velocity, one can use a coordinate system where the charge q1 is at x=y=z=0 with a velocity v = 0.
Additionally the coordinate system will be chosen such that all elements are in the x-y plane, then no component is created in direction of the z-axis. The resulting forces are all in the x-y plane.
\\
\\
The vector formula with q2 at $r_x$  and  $r_y$ with $ \textbf{v} = v_x $ ; $v_y=v_z=0$  reduces to
\\
\\
\begin{equation}
\textbf{$F_1$}=\dfrac{q_1q_2}{10^7\vert\textbf{r}\vert^3}*
\begin{Bmatrix}
\begin{pmatrix}r_x \\ r_y\\0\end{pmatrix}+
\begin{pmatrix}0 \\	r_y\dfrac{v_x^2}{c^2} \\0\end{pmatrix}
\end{Bmatrix}
\end{equation}
\\
\\
This is the force on $q_1$.
\\
\\
The force on $  q_2$ has the same magnitude but opposite direction.
One can see that the first vector term is a central force between the two charges.
The second term creates a moment (torque) on the two charges, and with that does not add kinetic energy to the charges. The dynamic forces are opposite in direction but no central forces. 
\\
\\
With that the third Axiom from Newton has to be modified. Each force creates immediately the same counterforce in exactly the same opposite direction. This is valid for all central forces, like static electric forces, and gravitational forces. This is as before, new is now:
For charges there is a dynamic force that creates a rotational moment. The dynamic forces on the the charges have the same amount, opposite direction, the moment arm is the length from its present location to the 
\\
\\
At the point of closest approach in the formula (12) for $r_x = 0$ the moment arm is zero (0). 
This is always the case, if charge 2 is circling around charge 1. Then no moment is created, remaining is this dynamic force that adds to the central force of the static electric field.
According the previous formula the total force that acts against the rotation force is just double of the static force, when the speed reaches c.
\\
\\
Different simulations were done using this formula.        					
\\
\\
Presently the most impressive is the comparison of electron beam deflection, where the beam passes through opposite charged fixed spheres.
The result is, that the x-y path is basically identical with a calculation that uses the Lorentz-Transform from Special Relativity.
It was found by numerical simulations that a maximum deviation of 12,5\% appeared when the kinetic emergies are in the range of their rest mass $m*c^2$. If you go to lower or higher kinetic energies, then the two path calcuculations are nearly identical. Of course the calculated needed time for an electron, due to the time dilation in SR, is different. The time in SR is longer according the Gamma factor, thus allowing the the static electric force alone to bend the beam accordingly.
\\
\\
A further simulation calculation was done to expand the Bohr Atom Model with these dynamic forces.
This simulation calculation was done for all atoms, but considered in a first step only one electron around the kernel. Also the internal electric moment of electrons and protons were not considered in this first approach. A comparison was made to the existing Bohr Model. Basically no major difference can be observed for atoms with up to 30 protons. For atoms with 50 or more protons, the difference between the Classical Bohr Model and the extended model (that included the dynamic forces) show big differences. These differences could not be verified, as in all tables of ionisazion energies we looked at, we did not find experimental data for large atoms (Z>50) for their ionisation energy for the first electron around the kernel, when all other electrons are stripped off.
\\
\\
Another result seems interesting. In this simulation model a strange effect could be observed. With increasing the energy of the circling electron around the kernel, a lower limit for the radius showed up, with minor differences for different atoms. Is this the corresponding to the desnity of pulsars with their high density ?  The order of magnitude seems to be equal. And by the way, how was the density of pulsars determined ?  By Quantum Theory ? EOS; Equation Of State ? Some further points to check.
\\
\\



\begin{thebibliography}{100}
\bibitem [1]{Jackson}   Classical Electrodynamics,  3rd Edition, John David Jackson
\end{thebibliography}

\end{document}