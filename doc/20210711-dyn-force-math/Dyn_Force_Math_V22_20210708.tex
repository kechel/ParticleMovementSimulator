\documentclass[10pt,titlepage]{article}
\usepackage[utf8]{inputenc}
\usepackage[T1]{fontenc}
\usepackage{lmodern}
\usepackage[english]{babel}

\usepackage[a4paper]{geometry}
\usepackage{amsmath}
\usepackage{amsfonts}
\usepackage{amssymb}
\usepackage{amstext}
\usepackage{mathtools}
\usepackage{scrpage2}
\pagestyle{scrheadings}
\clearscrheadfoot

\ifoot{\copyright \ Ottmar Kechel }
  \ofoot{\today}
\cfoot{ Page \pagemark}


\begin{document}

\author{Ottmar Kechel}
\date{\today}


\large  {\LARGE{Theory of Force between Moving Elementary Charges}}
\\\\
Author: Ottmar Kechel        
\\\\
Version No.22 dated \today  \\\\
\tableofcontents
\newpage
\section{Short Summary}
Alternate Theory to Relativity
\\\\
Using Physics of 1894 (eighteen hundred ninety four)\\
without limiting the velocity of particles to c,\\
it is possible to derive a theory that can explain many observations that otherwise can only be explained by the Theory of Relativity.\\
\textbf{Key is the Lorentz Force.} \\
This first approach can explain (with derived formulas) relevant physical attributes of rotating charged elementary particles (e.g. atoms with one electron), like potential energy, kinetic rotation energy, angular moment, frequency.\\
It leads to an explanation and calculation of a neutron, giving the rotation radius, that exactly fits the expected value, today gained by scattering experiments.\\
It leads to the assumption that the Lorentz Force is one cause for gravity, as the kinetic rotation energy caused by the Lorentz force is nearly identical (delta 1.4\%) with the mass energy that differentiates the mass of a neutron from the mass of an electron plus proton.
Also the attempt to explain photons by rotating two charged particles seems to be successful, it fits exactly with the new Atom Model.




%%%%%%%%%%%%%%%%%%%%%
%%%%%%%%%%%%%%%%%%%%%
%%%%%%%%%%%%%%%%%%%%%
\section{Introduction}
Heinrich Hertz published in 1894 his book "Untersuchung Über Die Ausbreitung Der Elektrischen Kraft" [2]. Here he discussed several possible ways, how the electric force is acting. The first alternative, the direct immediate action, without any intermediate transmission was neglected, though there was no strong argument, no experiment was made, to exclude this possibility. \\
In the study of electromagnetic emission in the universe, especially from Neutron stars, one finds effects that can not be explained with today's physics. There seems to be a chance to come to sounding explanations, if one follows this first discussed path by Heinrich Hertz though it was dismissed.\\
This has as consequence, that there is no velocity limit for particles or electromagnetic waves (photons). Of course the emission velocity of photons from the source is c (as in emission theory).
\\\\
\textbf{This paper is a test or an experiment, 
\\
how far one can get using this starting point }\\\\
The result is in the eyes of the author amazing, it even leads to questions like: \\\\
Is the photon frequency exactly half of atom rotating frequency-difference ?\\
Is gravity a consequence of the Lorentz Force ?\\
Can a photon model be created that consists of two rotating particles ?\\
Can this photon model explain the particle and wave character ?\\


%%%%%%%%%%%%%%%%%%%%%
%%%%%%%%%%%%%%%%%%%%%
%%%%%%%%%%%%%%%%%%%%%
\section{Math of the new theory }

The force on charges is given by few equations.
The formulas are taken from 
\\John David Jackson [JDJ]; Classical Electrodynamics; Third Edition.
\\The F numbers refer to the Formula No. (F n.m) in that book.
\\\\
1. The Lorentz Force (Introduction F 1.3)


\begin{equation}
\textbf{F} = q(\textbf{E}+\textbf{v }\textbf{x}\textbf{ B})
\end{equation}
\\
2. The Coulomb Formula gives the static force between two charges $q_1$ and $q_2$ 
\\
\begin{equation}
\textbf{F} = \frac{1}{(4\pi \epsilon_0)} (q_1 \cdot q_2)\frac{\textbf{r}}{\vert\textbf{r}\vert^3}
\end{equation}
\\
As \ $c^2=\mu_0* \epsilon_0$ \ with \ $ \mu_0=4*\pi/10^7$ follows
\begin{equation}
\frac{1}{(4\pi \epsilon_0)}=\frac{c^2}{10^7}
\end{equation}\\
This replacemet will be used furtheron.
\\\\
so the Coulomb law has the following expression (Chapter 1  F 1.3)
\begin{equation}
\textbf{E} = \frac{ q \ c^2}{10^7}\frac{\textbf{x-x1}}{\vert\textbf{x-x1}\vert^3}
\end{equation}\\

%%%%%%%%%%%%%%%%%%%%%
%%%%%%%%%%%%%%%%%%%%%
%%%%%%%%%%%%%%%%%%%%%

3. The BIOT and Savart Formula (Chapter 5  F 5.5)
\\(By the way the same formula appears in chapter 11.10 in the Gauss unit system describing the B field of a passing charge.)\\
\\
\begin{equation}
\textbf{B} = k \ q \frac{\textbf{v} \times \textbf{x}}{\vert\textbf{x}\vert^3} = \frac{q}{10^{7}} \ \frac{\textbf{v} \times \textbf{x}}{\vert\textbf{x}\vert^3}
\end{equation}
\\
The constant factor k is in SI  $[10^{-7}]$ with the units $[N/A^2]$ =  $[kg \ m/C^2]$
\\

%%%%%%%%%%%%%%%%%%%%%
%%%%%%%%%%%%%%%%%%%%%
%%%%%%%%%%%%%%%%%%%%%

\subsection{Warning in the book "CLASSICAL ELECTRODYNAMICS" from John David Jackson}
Directly behind the B-field equation a warning is included in the book [1] consisting of three objections

"... But this expression is time dependent and furthermore is valid only for charges whose velocities are small compared to that of light and whose accelerations can be neglected .  ..."
\\\\
a) But this expression is time dependent is not valid for two charges circling around their center of mass.
\\ Here one can derive the force without any time dependence for a two particle system, as it is shown later on in this New Theory.
\\\\
b) In the New Theory the speed of light is no limit for particles, so that this limitation is not valid in the New Theory. 
\\
\\
c) For the acceleration the present interpretation is also not sufficient, as it is known, that the rotation of electrons in atoms around the kernel cannot be correctly interpreted by the present theory. The electrons are accelerated continuously and should emit electromagnetic waves. They do not ! So the present rule, that accelerated charges create electromagnetic waves needs a refinement. 
\\

%%%%%%%%%%%%%%%%%%%%%
%%%%%%%%%%%%%%%%%%%%%
%%%%%%%%%%%%%%%%%%%%%

\section{Combine Biot Savart formula \\with Coulomb force and Lorentz force}
Looking at two charges $q_1$ and $q_2$ rotating around their center of mass
\\
with their vectorial distance \ \ \ ${\bf r=r_2-r_1}$
\\
their vectorial relative velocity ${ \bf v = v_2-v_1}$
\\
one gets the following equations\\\\
for the electric field at the location of $q_1$
\begin{equation}
\textbf{E} = \frac{ q_2 \ c^2}{10^7}\frac{\textbf{r}}{\vert\textbf{r}\vert^3}
\end{equation}\\
for the B field according the Biot Savart formula at the location of $q_1$
\begin{equation}
\textbf{B} = \frac{q_2}{10^7} \frac{\textbf{v} \times \textbf{r}}{\vert\textbf{r}\vert^3}
\end{equation}
\\\\
Inserting these two fields into the Lorentz force formula with $q_1$ 
\begin{equation}
\textbf{F} = q_1 \ (\textbf{E}+\textbf{v}\times \textbf{ B})
\end{equation}
\\
results in
\begin{equation}
\textbf{F} = \frac{{q_1} \ {q_2}}{{10^7}}( \frac{c^2\textbf{r}}{\vert\textbf{r}\vert^3}+\textbf{v}\times ( \frac{\textbf{v} \times  \textbf{r}}{\vert\textbf{r}\vert^3}))
\end{equation}\\
This is equal to

\begin{equation}
\textbf{F} = \frac{q_1 \ q_2}{10^7\vert\textbf{r}\vert^3}  (c^2\textbf{r}+\textbf{v} \times ({\textbf{v} \times  \textbf{r}}))
\end{equation}\\
or
\begin{equation}
\textbf{F} = \frac{q_1 \ q_2 \ c^2}{10^7\vert\textbf{r}\vert^3}  (\textbf{r}+\frac{1}{c^2}(\textbf{v}\times( {\textbf{v} \times  \textbf{r}}))
\end{equation}
\\
or
\begin{equation}
\textbf{F} = \frac{q_1 \ q_2 \ c^2}{10^7\vert\textbf{r}\vert^3}  (\textbf{r}+(\frac{\textbf{v}}{c}\times( {\frac{\textbf{v}}{c}    \times {\textbf{r}}}))
\end{equation}
\\\\
From these equations follows
\\a) the force is always proportional to the product of the charges (for static and dynamic forces)
\\b) the force decreases always with the square of the distance (for static and dynamic forces)
\\c) there are always several components included, the first expression $c^2\textbf{r}$ in the brackets corresponds to the well known static force between the charges, the second expression represents the dynamic force, what usually is called the magnetic influence. This is velocity dependent. 
\\d) the first cross product generates a force that is always perpendicular to the momentary speed. That is equal to the rule that a moving charge in a magnetic field does not gain any energy, it is only deflected perpendicular to its momentary velocity vector.
\\e) The dynamic force is zero whenever the velocity \textbf{v } and radius vector \textbf{r } are parallel, and this can only be if they are on the same line to each other.
\\\\
\subsection{Transform cross product to sum of scalar products}
Now one can transform the two vector products according the mathematical identity
\begin{equation}
\textbf{a }\times (\textbf{ b } \times  \textbf{c })=\textbf{ b }(\textbf{a }\textbf{c }) - \textbf{c }(\textbf{a }\textbf{ b })
\end{equation}
\\
and gets
\begin{equation}
\textbf{F} = \frac{q_1 \ q_2}{10^7 \ \vert\textbf{r}\vert^3}  (c^2\textbf{r}  +   \textbf{ v }(\textbf{v }\textbf{r }) - \textbf{r }(\textbf{v }\textbf{ v }))
\end{equation}
\\\\
With that it is shown, that the forces between two charges moving relative to each other can be calculated with formula (13). The result is identical to a calculation where at first the B field is determined then the resulting force calculated.
But this expression does not include anymore a B-field expression. The calculation of a B field is not necessary.
The dynamical force between two charges is calculated directly. Anyhow the question arises, what B-field would be the correct one. Here you have two choices, either calculate the B-field of charge 1 acting on charge 2 or vice versa. Both calculations result in completely different B-fields. Only the resulting force is identical. So it only can be concluded, that a B-field does not exist. It is obviously only a helpful model, but does not reflect an existing field in between the charged particles.
The resulting force is only a direct force between the charges that move relative to each other. The resulting dynamical force is always in the plane given by the velocity vector and the distance vector. The direction is always the same as with the static force, two bodies with the same charge polarity repel each other (the static plus the dynamic force). If the charges have different polarities the particles attract each other more than only by the static force.
\\
One can write now the whole equation using vectors considering two charges
$q_1$ and $q_2$

\begin{equation}
\textbf{$\textbf{F}_1$}=\dfrac{c^2}{10^7}\dfrac{q_1 \ q_2}{\vert\textbf{r}\vert^3} \ 
\begin{Bmatrix}
\begin{pmatrix}r_{2x}-r_{1x}\ \\ r_{2y}-r_{1y}\\r_{2z}-r_{1z}\end{pmatrix}+
\begin{pmatrix}(r_{2x}-r_{1x})\dfrac{\textbf{v}^2}{c^2}\\ \\(r_{2y}-r_{1y})\dfrac{\textbf{v}^2}{c^2}\\ \\(r_{2z}-r_{1z})\dfrac{\textbf{v}^2}{c^2}\end{pmatrix}-
\begin{pmatrix}(v_{2x}-v_{1x})\dfrac{\textbf{v} \ \textbf{r}}{c^2}\\ \\(v_{2y}-v_{1y})\dfrac{\textbf{v} \ \textbf{r}}{c^2}\\ \\(v_{2z}-v_{1z})\dfrac{\textbf{v} \ \textbf{r}}{c^2}\end{pmatrix}
\end{Bmatrix}
\end{equation}\\
with  \begin{equation}
\textbf{v}^2=(v_{2x}-v_{1x})^2+(v_{2y}-v_{1y})^2+(v_{2z}-v_{1z})^2
\end{equation}  
\\and \begin{equation}
\textbf{v \ r}=(v_{2x}-v_{1x}) \ (r_{2x}-r_{1x})+(v_{2y}-v_{1y}) \ (r_{2y}-r_{1y})+(v_{2z}-v_{1z}) \ (r_{2z}-r_{1z})
\end{equation}
\\and   \begin{equation}
\vert\textbf{r}\vert^3=((r_{2x}-r_{1x})^2+(r_{2y}-r_{1y})^2+(r_{2z}-r_{1z})^2 \ \ )^{3/2}
\end{equation}
\\
\\
If you select the coordinate system such that charge 1 is \\ located at r=(0,0,0) with \\ velocity v=(0,0,0)\\
you get a somewhat simpler form for the equation giving the momentary force acting on charge 1.
\\
\\
\begin{equation}
\textbf{$\textbf{F}_1$}=\frac{c^2}{10^7}\dfrac{q_1 \ q_2}{\vert\textbf{r}\vert^3} \ 
\begin{Bmatrix}
\begin{pmatrix}r_{2x} \\ r_{2y}\\r_{2z}\end{pmatrix}+
\begin{pmatrix}r_{2x}\dfrac{v_2^2}{c^2} \\	r_{2y}\dfrac{v_2^2}{c^2} \\r_{2z}\dfrac{v_2^2}{c^2}\end{pmatrix}-
\begin{pmatrix}v_{2x}\dfrac{v_2 \ r_2}{c^2} \\	v_{2y}\dfrac{v_2 \ r_2}{c^2} \\v_{2z}\dfrac{v_2 \ r_2}{c^2}\end{pmatrix}
\end{Bmatrix}
\end{equation}
\\\\

Looking at a circular movement with radius (distance) $r=r_{2y}$ and perpendicular velocity $v=v_{2x}$ and
and all other components $=0$ one gets
\\
\\
\begin{equation}
\textbf{$\textbf{F}_2$}=\frac{c^2}{10^7}\dfrac{q_1 \ q_2}{\vert\textbf{r}\vert^3} \ 
\begin{Bmatrix}
\begin{pmatrix}0 \\ r_{2y}\\0\end{pmatrix}+
\begin{pmatrix}0 \\	r_{2y}\dfrac{v_{2x}^2}{c^2} \\0\end{pmatrix}
\end{Bmatrix}
\end{equation}
\\
\\
This leads to the basic formula for the rotation with constant radius and perpendicular velocity.
With the amount of the constant radius $r$ and the amount of the constant perpendicular velocity $v$ one gets
\\\\
\begin{equation}
F_2 =\frac{c^2}{10^7}\dfrac{q_1 \ q_2}{\vert\textbf{r}\vert^2}  (1+ \frac{v^2}{c^2})
\end{equation}
\\
This is the attracting force between two rotating charges with opposit polarity.
\\
\\
Both forces are central forces.
The rotation force is just double of the static force, when the relative speed is c.
\\
\\
%%%%%%%%%%%%%%%%%%%%%
%%%%%%%%%%%%%%%%%%%%%
%%%%%%%%%%%%%%%%%%%%%

\section{The electric forces}

%%%%%%%%%%%%%%%%%%%%%
%%%%%%%%%%%%%%%%%%%%%
%%%%%%%%%%%%%%%%%%%%%
        					
\subsection{Cherged particle beam deflection according \\Special Relativity versus Coulomb and Lorentz force}
The beam deflection (also valid for electrons) was numerically calculated, where the beam passes through opposite charged fixed spheres, with two different theories\\\\
For both calculations the same x-velocity energy was chosen with identical acceleration voltage, and identical voltage between the deflection spheres. The starting point was chosen in the middle between the charged spheres. So for the x-velocity the charge is multiplied with the acceleration voltage giving for both models the same kinetic energy.
Then the following calculations were done to determin the y-elevation of an electron beam.
\\\\
A: numerical step by step calculation according Special Relativity (SR)\\ and Coulomb force only
\\\\
B: numerical step by step calculation using Coulomb and Lorentz force\\ without c as limiting velocity (no relativistic calculation).
\\\\
The result is, that the y-deflection is basically identical \\a) calculated according Special Relativity (SR) using the Coulomb force only\\b) calculated with the new theory using the Coulomb and Lorentz-force.
\\\\
It was found that a maximum deviation between the two theories of 12,5\% appeared, when the kinetic energies in x-direction are in the range of their rest mass $m \ c^2$, what is equivalent to a Gamma factor of 2 in SR. If you go to lower or higher kinetic energies, then the two y-deflection calculations are (nearly) identical.
\\\\
Of course the calculated needed time for a cgarged particle (e.g. electron) to run along a given path, due to the time dilation in SR, is different. The time in SR is longer according the Gamma factor, thus allowing the static electric force alone to bend the beam accordingly. This time dilation is not given in the new theory using the Lorentz force. The stronger bending (compared with the classical calculation considering the static Coulomb force only) is due to the dynamic force or one even could say due to the "magnetic" interaction of the charges due to the Lorentz Force.
\\\\

%%%%%%%%%%%%%%%%%%%%%
%%%%%%%%%%%%%%%%%%%%%
%%%%%%%%%%%%%%%%%%%%%

\subsection{Static force equation extended with dynamic Lorentz Force}
Now one can derive an extended atom model using the static Coulomb Force and the dynamic (velocity dependent)  Lorentz Force.
\\
\\
If you take formula (18) and set $r_{2x}$ to zero you have the attracting force of charge $q_2$ to $q_1$ for a circled movement. (assuming  $q_2$<0 ; $q_1$>0).
Adding now the radial acceleration $v^2/r$ you come to an extended Atom-Model that additionally to the Bohr-Atom model not only considers the static force and radial acceleration with its resulting force but also the dynamic velocity dependent Lorentz force. Replace $q_1$ by $n_p \ (-q_e)$ i.e. the charge of a proton multiplied the number $(n_p)$ of protons in one atom kernel and $q_2$ by $q_e$ (the charge of one electron)
leads to the following equation that must be zero (=0).
\\
The different values of the radius vectors have to be considered. For the electric forces $r$ is the distance  between the particles, for the centrifugal force it is the rotation radius $r_r$ around the center of mass.
\begin{equation}
\textbf{$\textbf{F}_1$}=\frac{c^2}{10^7}\dfrac{q_e \ n_p \ (-q_e)}{r^3} \ 
\begin{Bmatrix}
\begin{pmatrix}0 \\ r\\0\end{pmatrix}+
\begin{pmatrix}0 \\	r\dfrac{v^2}{c^2} \\0\end{pmatrix}
\end{Bmatrix} +
m_e  \ 
\begin{pmatrix}0 \\
	\dfrac{v^2}{r_r} \\0\end{pmatrix}=0
\end{equation}
\\or
\begin{equation}
\textbf{$\textbf{F}_1$}=\frac{c^2}{10^7}\dfrac{q_e \ n_p \ (-q_e)}{r^2} \ 
\begin{Bmatrix}
\begin{pmatrix}0 \\ 1\\0\end{pmatrix}+
\begin{pmatrix}0 \\	\dfrac{v^2}{c^2} \\0\end{pmatrix}
\end{Bmatrix} +
m_e  \ 
\begin{pmatrix}0 \\	\dfrac{v^2}{r_r} \\0\end{pmatrix}
=0
\end{equation}
\\($m_e$ is the mass of an electron; $q_e$ the charge of an electron; $n_p$ is the number of protons in the kernel)
\\\\
Here one can see the influence of the Lorentz force. If you set the Coulomb force to 1 the Lorentz force contributes $v^2/c^2$.
so the force is proportional to the sum
\begin{equation}
\textbf{${F}_1$} \propto (1+\frac{v^2}{c^2})
\end{equation}
$F_1$ is parallel to the radius vector.
%%%%%%%%%%%%%%%%%%%%%
%%%%%%%%%%%%%%%%%%%%%
%%%%%%%%%%%%%%%%%%%%%

\subsection{A limiting radius $r_g$}
One first result from this extended Atom Model seems interesting. In the new model, due to the Lorentz Force, appears a special rotation radius $r_g$. The model shows a value of  $r_g=q_1 \ q_2/(10^7 \ m_1)$. \\This is for the Hydrogen atom $2,8180 \ 10^{-15}$ m and is about 18.000 times smaller than the rotation radius of the ground state electron in Hydrogen. This radius $r_g$ can never be achieved, as then the kinectic energy would go to infinity. So there is a range around $r_g$ that is not achieveable. The kinetic energy would be higher than the contained potential energy of two charged particles that are separated at infinite large distance.
\\\\

%%%%%%%%%%%%%%%%%%%%%
%%%%%%%%%%%%%%%%%%%%%
%%%%%%%%%%%%%%%%%%%%%

\subsection{Adding magnetic dipol to magnetic dipol force}
Looking at the forces between two oppositly charged particles there is one force missing. Additionally to the static force and Lorentz Force one has to consider the magnetic force, as the electron and the proton/kernel of an atom have a magnetic moment. They behave like small magnets and can be described by their magnetic dipol moment.\\ 
\\\\
The force between magnetic dipols ($\mu_1 \text{ and } \mu_2$) is dependent on \\the distance d that is equal to the rotation radius corrected with with the mass factor $(1+m1/m2)$ and on \\ the orientation of the magnetic moments (either parallel or antiparallel) and the orientation relative to the rotation axis (either parallel of perpendicular).
\\\\
A) If the magnetic moments are perpendicular to the rotation axis, the force with rotation radius $r_1$ is
\begin{equation}
\ F=\frac{6 \ \mu_1 \ \mu_2}{10^7 \ r_1^4 \ (1+m_1/m_2)^4  }
\end{equation}
The force is attracting if $\mu_1$ is parallel to $\mu_2$; distracting if antiparallel.
\\\\
B) If the rotation axis is parallel to the magnetic moments, the force is
\begin{equation}
\ F=\frac{3 \ \mu_1 \ \mu_2}{10^7 \ r_1^4 \ (1+m_1/m_2)^4  }
\end{equation}
If the magnetic moments $\mu_1$ and $\mu_2$ are parallel the force is distracting, if antiparallel attracting each other.
\\ Distracting is assumed to be the normal case for atoms, as then the contained energy has a minimum. There is a possibility that the orientation of the the two magnetic moments are antiparallel, that leads to slightly higher energies. A switch from antiparallel to parallel orientation occurs at the spin flip (known from the Hydrogen spin flip resulting in the important spin flip frequency that is known with a very high accuracy).
\\\\

%%%%%%%%%%%%%%%%%%%%%
%%%%%%%%%%%%%%%%%%%%%
%%%%%%%%%%%%%%%%%%%%%

\section{The complete force equation \\ for two opposit charged particles \\rotating around their center of mass}

%%%%%%%%%%%%%%%%%%%%%
%%%%%%%%%%%%%%%%%%%%%
%%%%%%%%%%%%%%%%%%%%%
\subsection{Some basics for a rotating two particle system}
For a rotating two perticle system the following condition exists using the center of mass as reference point

\begin{equation}
\ \frac{r_1}{r_2}=\frac{v_1}{v_2}=\frac{m_2}{m_1}
\end{equation}
\\
The distance between the two particles can be expressed by the rotation radius with
\begin{equation}
d=r_1+r_2=r_1+r_1 \ \frac{m_1}{m_2}=r_1 \ (1+\frac{m_1}{m_2})
\end{equation}
\\
The same factor applies for the velocity difference between two rotating particles needed for the Lorentz force
(using only positive values). 
\begin{equation}
v=v_1+v_2=v_1 \ (1+\frac{m_1}{m_2})
\end{equation}
\\\\
When you know the kinetic rotation energy $E_{kin1}$ of $m_1$, the total kinetic rotation energy of $m_1$ and $m_2$ is given by
\begin{equation}
\ E_{kin}=\frac{1}{2} \ m_1 \ (1+\frac{m_1}{m_2}) \ v_1^2=E_{kin1}(m1) \ (1+\frac{m_1}{m_2})
\end{equation}
The same factor is used to calculate the total angular moment from the angular moment of particle 1.
\\\\

%%%%%%%%%%%%%%%%%%%%%
%%%%%%%%%%%%%%%%%%%%%
%%%%%%%%%%%%%%%%%%%%%

\subsection{Force equation}
(For all following sections the variable $r$ is the rotation radius, otherwise noted.)
\\
\\
Now we have all forces that are acting on two opposit charged particles.
\\
The radial velocity dependent centrifugal force is compensated by \\
a) the static force \\
b) the dynamic velocity dependent Lorentz Force \\
c) the magnetic dipol to dipol force \\(positive if the dipol orientations are antiparallel, negative if they are parallel)
\\\\
Using only positive absolute values for the charges and parallel magnetic moments $(q_1>0 ; q_2>0 ; \mu_1>0 ; \mu_2>0)$ and $r_1$ as the rotation radius of $m_1$ the formula is

\begin{equation}
\ m_1 \ \frac{v_1^2}{r_1}=\frac{c^2 \ q_1 \ q_2}{10^7 \ (1+\frac{m_1}{m_2})^2 \ r_1^2}+\frac{q_1 \ q_2 \ v_1^2}{10^7 \ r_1^2}-\frac{3 \ \mu_1 \ \mu_2}{10^7 \ (1+\frac{m_1}{m_2})^4 \ r_1^4}
\end{equation}
\\
\\
This formula describes the forces on two opposit charged particles on circled path around the center of mass for parallel magnetic moments. It is interesting to see, that the Lorentz force is not dependent on the mass relation, as this factor is identical for the velocity and radius, so it cancels out.
\\\\


%%%%%%%%%%%%%%%%%%%%%
%%%%%%%%%%%%%%%%%%%%%
%%%%%%%%%%%%%%%%%%%%%

\subsection{Squared velocity equation}
Multipying the force equation with $r_1/m_1$ results in a summation of squared velocities.
\begin{equation}
\label{velocity_equation}
 v_1^2=\frac{c^2 \ q_1 \ q_2}{10^7 \ m_1 \ (1+\frac{m_1}{m_2})^2 \ r_1}+\frac{q_1 \ q_2 \ v_1^2}{10^7 \ m_1 \ r_1}-\frac{3 \ \mu_1 \ \mu_2}{10^7 \ m_1 \ (1+\frac{m_1}{m_2})^4 \ r_1^3}
\end{equation}
\\
setting  $$r_g=\frac{q_1 \ q_2}{10^7 \ m_1}$$


\begin{equation} v_1^2=\frac{c^2 \ r_g}{(1+\frac{m_1}{m_2})^2 \ r_1}+\frac{r_g \ v_1^2}{r_1}-\frac{3 \ \mu_1 \ \mu_2 \ r_g}{q_1 \ q_2 \ (1+\frac{m_1}{m_2})^4 \ r_1^3}
\end{equation}

\begin{equation} v_1^2 \ (1-\frac{r_g}{r_1})=\frac{c^2 \ r_g}{(1+\frac{m_1}{m_2})^2 \ r_1}-\frac{3 \ \mu_1 \ \mu_2 \ r_g}{q_1 \ q_2 \ (1+\frac{m_1}{m_2})^4 \ r_1^3}
\end{equation}

\begin{equation} v_1^2 \ (\frac{r_1}{r_g}-1)=\frac{c^2}{(1+\frac{m_1}{m_2})^2}-\frac{3 \ \mu_1 \ \mu_2}{q_1 \ q_2 \ (1+\frac{m_1}{m_2})^4 \ r_1^2}
\end{equation}
solving for $v_1^2$
\begin{equation} v_1^2=\frac{c^2}{(1+\frac{m_1}{m_2})^2 \ (\frac{r_1}{r_g}-1)}-\frac{3 \ \mu_1 \ \mu_2}{q_1 \ q_2 \ (1+\frac{m_1}{m_2})^4 \ r_1^2 \ (\frac{r_1}{r_g}-1)}
\end{equation}
\\
This can be interpreted as the summation/subtraction of two squared velocities
for parallel magnetic moments
\begin{equation}
v_1^2= v_C^2 - v_M^2 
\end{equation}
\\
for antiparallel magnetic moments
\begin{equation}
v_1^2= v_C^2 + v_M^2 
\end{equation}
\\
$v_C$ caused by the Coulomb force and Lorentz force\\
$v_M$ caused by the magnetic moment force and the Lorentz force

%%%%%%%%%%%%%%%%%%%%%
%%%%%%%%%%%%%%%%%%%%%
%%%%%%%%%%%%%%%%%%%%%


\subsection{Energy equation derived from velocity equation}

Multiplying the velocity equation (\ref{velocity_equation}) with $1/2 \ m_1 \ (1+m_1/m_2)$ leads to the summation of the kinetic rotation energies for both particles\\\\ $E_{kin}=$
\begin{equation} \frac{1}{2} \ m_1 \ (1+\frac{m_1}{m_2}) \ v_1^2=
\frac{c^2 \ q_1 \ q_2}{2 \ 10^7 \ (1+\frac{m_1}{m_2}) \ r_1}
+\frac{q_1 \ q_2 \ (1+\frac{m_1}{m_2}) \ v_1^2}{2 \ 10^7 \ r_1}
-\frac{3 \ \mu_1 \ \mu_2}{2 \ 10^7 \ (1+\frac{m_1}{m_2})^3 \ r_1^3}
\end{equation}
\\

%%%%%%%%%%%%%%%%%%%%%
%%%%%%%%%%%%%%%%%%%%%
%%%%%%%%%%%%%%%%%%%%%

\section{Kinetic energy, angular moment, and frequency \\dependent on rotation radius  $r_1$ only.}
The following abbreviation is used
\\
\begin{equation}
\ r_g=\frac{q_1 \ q_2}{10^7 \ m_1}
\end{equation}
\\\\
With the derived basic equations it is possible to calculate the kinetic energy, angular moment, and rotation frequency, all only dependent on the radius $r_1$, when the masses, the charges, and the magnetic moments are known. From basic nature constants only the masses, the charges,, and the speed of light is used. The Planck Constant h is not used.
\\\\

%%%%%%%%%%%%%%%%%%%%%
%%%%%%%%%%%%%%%%%%%%%
%%%%%%%%%%%%%%%%%%%%%


\subsection{Kinetic rotation energy}
The kinetic rotation energy for a two particle system with magnetic moments parallel, inserting $\mu_1$ and $\mu_2$ as positive values is (for parallel spin)
\\
\begin{equation}
\ E_{kin}=\frac{m_1 \ c^2}{2 \ (1+\frac{m_1}{m_2}) \   |(\frac{r_1}{r_g}-1)|}    \ (1-\frac{3 \ \mu_1  \mu_2}{c^2 \ q_1 \ q_2 \ (1+\frac{m_1}{m_2})^2 \ r_1^2})
\end{equation}
For antiparallel spin the energy portion of the magnetic moment has to be added (1+....).
\\\\

%%%%%%%%%%%%%%%%%%%%%
%%%%%%%%%%%%%%%%%%%%%
%%%%%%%%%%%%%%%%%%%%%

\subsubsection{Comparison with the Bohr Model}
If one sets \\
$m_1 << m_2$, and 
$\mu_1$ or $\mu_2$ = 0, and 
use ($r_1/r_g$) instead of ($r_1/r_g-1$)\\
then one arrives at the kinetic energy known from the Bohr model.\\
\begin{equation}
\ E_{kin}(Bohr)=\frac{m_1 \ c^2}{2 \   (\frac{r_1}{r_g})} = \frac{m_1 \ c^2 \ r_g}{2 \ r_1}= \frac{m_1 \ c^2 \ e^2}{2 \ 10^7 \ m_1 \ r_1}= \frac{e^2}{8 \ \pi \ \epsilon_0 \ r_1}
\end{equation}
\\\\

%%%%%%%%%%%%%%%%%%%%%
%%%%%%%%%%%%%%%%%%%%%
%%%%%%%%%%%%%%%%%%%%%

\subsection{Potential energy}
With the integration of the Coulomb force one can calculate the difference of the potential energy coming from infinte distance to a smaller distance d (this is used in the Bohr model).
The total contained potential energy in a charged two particle system is finite, it has a fixed value lets call it $E_\infty$ . Then the potential energy at $r_1$ is.

\begin{equation} 
E_{pot}(r_1) = E_\infty - \frac{c^2 \ q_1 \ q_2}{10^7 \ (1+\frac{m_1}{m_2}) \ r_1}
\end{equation} 
with $r_1$ as rotation radius of $m_1$.
\\
This is only an approximation valid only for radii $r_1 \gg $ radius of the particles $m_1$ and $m_2$. It can never become negative.
\\\\
 When the two charges come together to form an atom the total energy must be the same
\begin{itemize}
	\item $E_\infty$ =
	\item potential energy $E_{pot}$ at radius $r_1$ plus
	\item kinetic rotation energy $E_{kin}$ at $r_1$ plus
	\item energy of the emitted photon $E_p$ for the transition of $m_1$ from infinity to $r_1$  
\end{itemize}   
(This follows from Conservation of energy).
\\\\
So one can write for the energy 
\begin{equation} 
\mid \Delta E_{pot} \mid =\frac{c^2 \ q_1 \ q_2}{10^7 \ (1+\frac{m_1}{m_2}) \ r_1}=E_{kin} (r_1) + E_p
\end{equation}
\\\\
%%%%%%%%%%%%%%%%%%%%%
%%%%%%%%%%%%%%%%%%%%%
%%%%%%%%%%%%%%%%%%%%%

\subsection{Angular moment}
The angular moment for a two particle system amounts for parallel spin to 
\\
\begin{equation}
 L=\frac{m_1 \ r_1 \ c}{\sqrt{|(\frac{r_1}{r_g}-1)|}} \ \sqrt{1-\frac{3 \ \mu_1 \ \mu_2}{c^2 \ q_1 \ q_2 \ (1+\frac{m_1}{m_2})^2 \ r_1^2}}\end{equation}
\\
\\

%%%%%%%%%%%%%%%%%%%%%
%%%%%%%%%%%%%%%%%%%%%
%%%%%%%%%%%%%%%%%%%%%

\subsection{Rotation frequency}
The frequency for parallel spin is given by
\begin{equation}
\ 2 \ \pi \ f=\frac{c}{(1+\frac{m_1}{m_2}) \ r_1 \ \sqrt{|(\frac{r_1}{r_g}-1)|}} \ \sqrt{1-\frac{3 \ \mu_1 \ \mu_2}{c^2 \ q_1 \ q_2 \ (1+\frac{m_1}{m_2})^2 \ r_1^2}}
\end{equation}
\\
For numerical analysis to calculate the radius for a given rotation frequency this formula can be transformed to
\begin{equation}
\label{eqraf}
\ r_1=\frac{c}{(1+\frac{m_1}{m_2}) \ 2 \ \pi \ f \ \sqrt{|(\frac{r_1}{r_g}-1)|}} \ \sqrt{1-\frac{3 \ \mu_1 \ \mu_2}{c^2 \ q_1 \ q_2 \ (1+\frac{m_1}{m_2})^2 \ r_1^2}}
\end{equation}
\\\\
The frequency equation is dependent on the energy equation and angular moment equation as
\\
\begin{equation}
\ \omega=2 \ \pi \ f=\frac{2 \ E_{kin}}{L}
\end{equation}
\\
\\
So we have now for a given pair of opposit charged particles equations that are only dependent on the rotation radius $r_1$.
\\\\

%%%%%%%%%%%%%%%%%%%%%
%%%%%%%%%%%%%%%%%%%%%
%%%%%%%%%%%%%%%%%%%%%

\section{Connection to photon and spectral lines}
Apply the formulas to atoms with their known and measured spectral lines.\\

%%%%%%%%%%%%%%%%%%%%%
%%%%%%%%%%%%%%%%%%%%%
%%%%%%%%%%%%%%%%%%%%%

\subsection{Rotation frequency and photon frequency}
Looking at an electron coming from infinity to the stable ground state radius a photon is created.
This photon has a frequency $f_p$ that is exact half the atom (electron and proton) rotating frequency $f_a$ .\\

\begin{equation}
\ f_p = 1/2  \  f_a
\end{equation}
This is corresponding to the reverse of ionization, starting from infinite radius.
For other transitions $f_a$ is accordingly only the difference between the two states. 
\\\\
Using now known ionization energies for the first electron (e.g. from NIST) it is possible to calculate the radius in ground state of various atoms circled by one electron. 
\\
From the calculated radius one can then calculate the kinetic rotation energy of the atom $E_{akin}$ and also the angular moment $L_a$.
\\

%%%%%%%%%%%%%%%%%%%%%
%%%%%%%%%%%%%%%%%%%%%
%%%%%%%%%%%%%%%%%%%%%

\subsection{Angular moment}
From the conservation of angular moment it follows that the created photon must have an angular moment $L_p$ of the same amount. (vector sum of angular moments must be zero). For the absolute values we have
\\
\begin{equation}
\ L_p = L_a
\end{equation}
This is corresponding to the ionization, starting from infinite radius.
For other transitions $L_a$ is accordingly only the difference between the two states. 

%%%%%%%%%%%%%%%%%%%%%
%%%%%%%%%%%%%%%%%%%%%
%%%%%%%%%%%%%%%%%%%%%

\subsection{Energy}
(The mass energies of the particles are not taken into consideration, as  they are are fixed, independent of their distance.)\\
Two masses with opposit charges have a fixed potential energy $E_\infty$ when they are infinitely far apart from each other. In the Bohr model this energy potential is set to zero, as the real absolute value is not known. With reducing the distance between the particles the Bohr energy is negative by that reducing the potential energy $E_\infty$. With setting the total energy at infinity to $E_\infty$ the energy remains positive. It is reduced when the radius gets smaller, the pair looses potential energy that is converted into
\\\\
a) kinetic rotation energy $E_{akin}(r)$ of the atom and 
\\\\
b) photon energy $E_p$
\\\\
So one can write 
\begin{equation}
\ E_\infty=E_{apot}(r\rightarrow \infty)=E_{apot}(r)+E_{akin}(r)+E_p
\end{equation}
All these energies are positive or zero, never negative.
\\Up to now $E_{apot}(r)$ and $E_p$ is not known. 
\\\\
Potential energy of atom:\\
For the potential energy one can make the assumption that for an atom in ground state with one electron in the shell, the kinetic energy is identical with the absolute potential energy.
 \begin{equation}
\ E_{apot}(r)=E_{akin}(r)
\end{equation}
That is most probably the reason, as no other reason seems to be imaginable why the electron in ground state reaches exact the radius to form stable atoms.
So one gets in ground state
\begin{equation}
\ E_\infty=2 \ E_{akin}(r)+E_p
\end{equation}
\\\\
Photon energy:\\
The emitted photon will get exactly half the kinetic energy the atom got. This follows immediately from the condition that $f_p$ is half of $f_a$, and the absolute values of the angular moments are identical.
\\
Taking now into considereation that the kinetic rotation energy of the photon is exactly half of the atoms rotation energy and that the potential photon energy equals exact the photons kinetic energy, one gets.
\begin{equation}
\ E_\infty=3 \ E_{akin}(r)+E_{pMT}
\end{equation}
The expression $E_{pMT}$ stands for additional energy portions the photon carries. The index M stands for the mass energy of the photon particles, the index T stands for the translational energy of the total photon mass, as it is ejected from its source with light speed c.
\\

%%%%%%%%%%%%%%%%%%%%%
%%%%%%%%%%%%%%%%%%%%%
%%%%%%%%%%%%%%%%%%%%%

\section{Calculation from using spectral lines}
With numerical calculations one can get from the spectral lines the radius of the electron. All other values Ekin, L can then be determined. \\
Also other transitions from shell n to m can be calculated if there is only one electron circling around the kernel.\\
If one looks at the transition for an ion, e.g. an atom that has already n electrons and the n+1th electron makes a transition form shell n+2 to n+1, the mass of the kernel has to be adjusted (add the mass of the electrons) with the mass of the already rotating electrons.\\
This calculation delivers exact frequency values compared with spectral lines.\\The photons have always half the frequency of the atom rotation frequency difference. \\\\


%%%%%%%%%%%%%%%%%%%%%
%%%%%%%%%%%%%%%%%%%%%
%%%%%%%%%%%%%%%%%%%%%

\section{Energy minimum link to Neutron}
With the help of the new theory it is possible to calculate the total energy contained in an electron proton pair. 
\\
The total energy content of an electron proton pair can be calculated with the derived formulas from potential and kinetic energy
\begin{equation} 
E(r_1) = E_\infty -  \frac{c^2 \ q_1 \ q_2}{10^7 \ (1+\frac{m_1}{m_2}) \ r_1}+\frac{m_1 \ c^2}{2 \ (1+\frac{m_1}{m_2}) \   |(\frac{r_1}{r_g}-1)|}    \ (1-\frac{3 \ \mu_1  \mu_2}{c^2 \ q_1 \ q_2 \ (1+\frac{m_1}{m_2})^2 \ r_1^2})
\end{equation} 
%%%%%%%%%%%%%%%%%%%%%
%%%%%%%%%%%%%%%%%%%%%
%%%%%%%%%%%%%%%%%%%%%
\\\\
The reference potential energy $E_\infty$ was selected such, that the potential energy equals the kinetic energy of a Hydrogen atom in ground state.
The result is shown in Fig.1.
\\
The graphic shows that the energy would grow to infinity at $r_g$, but also shows that there is a minimum at a radius smaller than $r_g$. The radius of this minimum is not influenced by the reference potential energy, it stays the same for any constant added to the potential energy. \\
Of interest is now the value of the radius, where one finds this minimum. It is at r = $1,8853.. \ 10^{-15}$ m, close to the expected radius of a neutron. Here it is known from latest scattering experiments, that the radius is in a range of about $1,8  \ 10^{-15}$m.\\\\
There is a second fact. Looking at the energy it is possible with this new theory to differentiate between the kinetic energy caused by the magnetic dipol moments $E_m$ and the kinetic energy caused by the static and Lorentz Force $E_L$. 
\\\\
Calculating the mass difference m(neutron) - m(electron) - m(proton) you arrive at $1,39422E-30$ kg. Converting this into mass energy according $E=m \ c^2$, one gets $0,782$ MeV.\\
Determining numerically with the new theory at the energy minimum the energy caused by static and Lorentz Force one arrives at  $E_L$=$0,771$ MeV.\\
This is only 1,35 \% smaller than the calculated value from the mass difference.
Now one can ask: \\
Is the energy caused by static and Lorentz Force underlying garavity ?\\
Is this mass ?\\
Is the Lorentz Force one reason for creating mass and gravity ?
\\\\
A third result is remarkable. With this new theory one can calculate the Lorentz force at this radius, where the total energy has a minimum. The calculation gives a value for the absolute velocity difference betweeen proton and electron  of about $12.15$ times c. In this theory the Lorentz force is $v^2/c^2$ stronger than the static Coulomb force. So the Lorentz force is about 147 times stonger than the Coulomb force.
\\\\
In nuclear physics it is known, that the Strong Interaction, the strong nuclear force (today described as one of the fundamental interactions) is about 137 times as strong as electromagnetism. This is in the same magnitude and range as the Lorentz force for a Neutron according to this new theory. The relative difference is less than 10 \%.\\
So the question arises: \\
\begin{itemize}
	\item Can the Strong Nuclear Force be explained with the Lorentz force ? 
\end{itemize}

\begin{figure}
	\centering
	\includegraphics [width=1\linewidth]{Neutron_20200313}
	\caption{}
	\label{fig:neutron20200308}
\end{figure}
\newpage

%%%%%%%%%%%%%%%%%%%%%
%%%%%%%%%%%%%%%%%%%%%
%%%%%%%%%%%%%%%%%%%%%

\section{Consequences of the new theory }
\textbf{
Starting point for this new theory is\\
\begin{itemize}
		\item c as the velocity of emitting photons from their source\\
		\item no limiting velocity for particles and photons \\
		\item a: the electric static (Coulomb) force  \\b: the dynamic (Lorentz) force \\c: the magnetic dipol to dipol force \\
		these three forces are acting immediately, without any retardation,\\
\end{itemize} 
In this new theory follows\\
\begin{itemize}
		\item the Lorentz Force is a cause for gravity\\
		\item then gravity is also acting immediately without any retardation.\\
\end{itemize} }

%%%%%%%%%%%%%%%%%%%%%
%%%%%%%%%%%%%%%%%%%%%
%%%%%%%%%%%%%%%%%%%%%

\section{The Photon model}
Using exact the same basic equations as for the atom with one electron one can set up a photon model consisting of:
\\\\
1. two rotating particles with mass $m_{p1}$ = $m_{p2}$\\
2. both have the same amount of charge\\
 that is exact the elementary charge e, 
so $q_{p1}   =  -q_{p2}  = e$
\\
3. both have the same amount of magnetic dipol moment\\
$ \mu_{ p1}= \mu_{p2}$ that are oriented parallel (leading to energy minimum).\\
4. The velocity around the center of mass of the two particles is
\begin{equation}
v_p=2 \ \pi \ c
\end{equation}
so that their relative speed is
\begin{equation}
v_{prel}=4 \ \pi \ c
\end{equation}
\\\\With $r_{p1}$ as rotation radius of the particles to the center of mass their distance is $2 \ r_{p1}$,
it follows that the photon frequency is
\begin{equation}\\\\
f_{p}=c/r_{p1}
\end{equation}
\\
With these assumptions we get the following\\
\\\\
%%%%%%%%%%%%%%%%%%%%%
%%%%%%%%%%%%%%%%%%%%%
%%%%%%%%%%%%%%%%%%%%%
\subsection{The force equations for a photon}
 
 Centrifugal force
 \begin{equation}
 F_c=(2 \ \pi \ c)^2 \ m_{p1}/r_{p1}
 \end{equation}
or as function of $f_p$
\begin{equation}
 F_c=(2 \ \pi)^2 \ c \ m_{p1} \ f_p
 \end{equation}\\
 The Coulomb force
 \begin{equation}
 F_q=\frac{c^2 \ e^2}{10^7 \ 2^2 \ r_{p1}^2}
 \end{equation}
or as function of $f_p$
  \begin{equation}
 F_q=\frac{e^2 \ f_p^2}{10^7 \ 2^2}
 \end{equation}\\
The Lorentz Force
\begin{equation}
F_L=\frac{e^2 \ (v_1-v_2)^2}{10^7 \ 2^2 \ r_{p1}^2}=\frac{(2 \ \pi \ e \ c)^2}{10^7 \ r_{p1}^2}
\end{equation} 
or as function of $f_p$
\begin{equation}
F_L=\frac{(2 \ \pi \ e)^2 \ f_p^2}{10^7}
\end{equation} \\
The magnetic force (from magnetic moments $\mu_{p1}=\mu_{p2}$)
\begin{equation}
F_m=\frac{3 \ \mu_{p1}^2}{10^7 \ 2^4 \ r_{p1}^4}
\end{equation}
or as function of $f_p$
\begin{equation}
F_m=\frac{3 \ \mu_{p1}^2 \ f_p^4}{10^7 \ 2^4 \ c^4}
\end{equation}
\\

 With the given assumptions for the photon and these formulas one can derive all following relations
 
 
 %%%%%%%%%%%%%%%%%%%%%
 %%%%%%%%%%%%%%%%%%%%%
 %%%%%%%%%%%%%%%%%%%%%
 
\subsection{Kinetic energy of a photon}
The kinetic rotation energy is (magnetic moments are parallel)
\\using $r_g=e^2/(10^7 \ m_{p1})$
\begin{equation}
\ E_{pkin}=\frac{m_{p1} \ c^2}{4 \ |(\frac{r_{p1}}{r_g}-1)|}    \ (1-\frac{3 \ \mu_{p1}^2}{c^2 \ e^2 \ 4 \ r_{p1}^2})
\end{equation}
\\\\
The total force equation can be written as\\
\begin{equation}
(2 \ \pi \ c)^2 \ m_{p1}/r_{p1}=\frac{e^2 \ c^2}{10^7 \ 2^2 \ r_{p1}^2}+\frac{(2 \ \pi \ e \ c)^2}{10^7 \ r_{p1}^2}-\frac{3 \ \mu_{p1}^2}{10^7 \ 2^4 \ r_{p1}^4}
\end{equation}
or as function of $f_p$ with \ $r_{p1}=c/f_p$
\begin{equation}
(2 \ \pi)^2 \ c \ m_{p1} \ f_p=\frac{e^2 \ f_p^2}{10^7 \ 2^2}+\frac{(2 \ \pi \ e)^2 \ f_p^2}{10^7}-\frac{3 \ \mu_{p1}^2 \ f_p^4}{10^7 \ 2^4 \ c^4}
\end{equation}
This leads to the total rotational kinetic energy
\begin{equation}\label{ph_Epkin}
E_{pkin}=\frac{c \ e^2 \ f_p}{4 \ 10^7} \ (1 + (4 \ \pi)^2-\frac{3 \ \mu_{p1}^2 \ f_p^2}{4 \ c^4 \ e^2})
\end{equation}
Here one can see the different influences of the three forces. The kinetic energy caused by the Lorentz Force is always $(4 \ \pi)^2$ larger than the kinetic energy caused by the static Coulomb force. The kinetic energy caused by the magnetic moments is dependent on the magnetic moments itself and the frequency.
\\\\
Another approach is possible, due to the expected rotation velocity of $ v_{p1}=2 \ \pi \ c$
\\
\begin{equation}
E_{pkin}= 1/2 \ (m_{p1} + m_{p2}) \ (2 \ \pi \ c)^2= m_{p1}(2 \ \pi \ c)^2
\end{equation}

\subsection{Particle mass energy of a photon}

so it follows immediately
\begin{equation}
m_{p1}=\frac{E_{pkin}}{(2 \ \pi \ c)^2}
\end{equation}
so the total particle mass of the photon is twice this value 
\begin{equation}
m_{p}=\frac{E_{pkin}}{2 \ (\pi \ c)^2}
\end{equation}
with that the photon mass energy is
\begin{equation}
\label{eqEpM}
E_{pM}=m_{p}*c^2=\frac{E_{pkin}}{2 \ (\pi)^2}
\end{equation}




%%%%%%%%%%%%%%%%%%%%%
%%%%%%%%%%%%%%%%%%%%%
%%%%%%%%%%%%%%%%%%%%%

\subsection{Potential energy of a photon}
As here the same relation exists for the potential energy as for an atom with one electron orbiting the kernel in ground state, it can be concluded that the potential energy equals the kinetic energy $E_{ppot} = E_{pkin}$.
So the sum of potential and kinetic rotation energy of a photon depending on the frequency is then with
\begin{equation}
E_{pkin}+E_{ppot}= 2 \ E_{pkin}
\end{equation}

\begin{equation}
E_{pkin+ppot}=\frac{c \ e^2 \ f_p}{2 \ 10^7} \ (1 + (4 \ \pi)^2-\frac{3 \ \mu_{p1}^2 \ f_p^2}{4 \ c^4 \ e^2})
\end{equation}

%%%%%%%%%%%%%%%%%%%%%
%%%%%%%%%%%%%%%%%%%%%
%%%%%%%%%%%%%%%%%%%%%

\subsection{Angular moment of a photon}
For the angular moment follows
\\
\begin{equation}
L_p=\frac{m_{p1} \ r_{p1} \ c}{\sqrt{|(\frac{r_{p1}}{r_g}-1)|}} \ \sqrt{1-\frac{3 \ \mu_{p1}^2}{4 \ c^2 \ e^2 \ r_{p1}^2}}\end{equation}
\\
With the kowledge of the photon mass the angular moment can also be calculated by
\begin{equation}
L_p=2 \ m_{p1} \ r_{p1} \ v_{p1}
\end{equation}
 or
\begin{equation}
L_p=4 \ \pi \ m_{p1} \ c^2/f_p
\end{equation}
\\\\

%%%%%%%%%%%%%%%%%%%%%
%%%%%%%%%%%%%%%%%%%%%
%%%%%%%%%%%%%%%%%%%%%

\subsection{Rotation frequency of a photon}
The frequency is given by
\begin{equation}
\ 2 \ \pi \ f_p=\frac{c}{2 \ r_{p1} \ \sqrt{|(\frac{r_{p1}}{r_g}-1)|}} \ \sqrt{1-\frac{3 \ \mu_{p1}^2}{4 \ c^2 \ e^2 \ r_{p1}^2}}
\end{equation}
\\\\
This last equation is dependent on the former two as
\\
\begin{equation}
\ \omega_p=2 \ \pi \ f_p=\frac{2 \ E_{pkin}}{L_p}
\end{equation}
\\
We have the same equation for the atom (kernel plus one electron)
\begin{equation}
\ \omega_a=2 \ \pi \ f_a=\frac{2 \ E_{akin}}{L_a}
\end{equation}
\\
From the relation that the frequency of the photon is half the frequency change of the atom-electron transition $f_p=1/2*f_a$
and due to the fact that the angular moment of the photon must have the same amount as the atomic angular moment change $L_p=L_a$, then the kinetic energies of the photon must be half of the atom kinetic energy change.
\\\\
From
\begin{equation}
f_p=1/2 \ f_a
\end{equation}
follows

\begin{equation}
E_{pkin}=1/2 \ E_{akin}
\end{equation}
\\\\
For the frequency also the following can be derived
\begin{equation}\label{key}
f_p^3=\frac{4 \ c^4}{3 \ \mu_{p1}^2} \ (e^2 \ f_p \ (1+(4 \ \pi)^2)-(4 \ \pi)^2 \ c \ 10^7 \ m_{p1}
)
\end{equation}
\\
Knowing the frequency the magnetic moment can be determined

\begin{equation}
\mu_{p1}^2=\frac{4 \ c^4}{3 \ f_p^3} \ (e^2 \ f_p \ (1+(4 \ \pi)^2)-(4 \ \pi)^2 \ c \ 10^7 \ m_{p1})
\end{equation}

%%%%%%%%%%%%%%%%%%%%%
%%%%%%%%%%%%%%%%%%%%%
%%%%%%%%%%%%%%%%%%%%%


\subsection{Translational photon particle energy}
Up to now only rotational energies and the potential were considered. Now one can look at the translational energy, as it is considered that the photon will be ejected from its source (the atom) with light speed c.\\
The additional translational energy $E_{pMT}$ considering the mass energy of the two photon particles is then
\begin{equation}
 E_{pMT} = 2 \ 1/2 \ m_{p1} \ c^2= m_{p1} \ c^2
\end{equation}\\
this is equal to
\begin{equation}
E_{pMT} = \frac{E_{pkin}}{(2 \ \pi)^2 }
\end{equation}\\
So as function of f this is
\begin{equation}
E_{pMT} = \frac{c \ e^2 \ f_p}{(4 \ \pi )^2 \ 10^7} \ \left[  1 + (4 \ \pi)^2-\frac{3 \ \mu_{p1}^2 \ f_p^2}{4 \ c^4 \ e^2})\right]
\end{equation}\\

%%%%%%%%%%%%%%%%%%%%%
%%%%%%%%%%%%%%%%%%%%%
%%%%%%%%%%%%%%%%%%%%%

\subsection{Translational photon energy by Lorentz mass}
If the rotation caused by the Lorentz force is causing gravitation / acts as mass, then one has to add this mass, lets call it the Lorentz mass. The energy itseld is contained in the kinetic rotation energy, but it acts as mass when it is accelerated to c.
\\\\
The Lorentz mass can be determind by the kinetic energy caused by the Lorentz force according formula \eqref{ph_Epkin}
\begin{equation}
E_{pkin}(Lorentz)=4 \ \pi^2 \ f_p \ c \ e^2/10^7
\end{equation}
and with that the additional Lorentz mass $m_L$ as  $m=E/c^2$
\begin{equation}
m_{pL}= \frac{4 \ \pi^2 \ e^2 \ f_p}{c \ 10^7}
\end{equation}
This addititional mass also has to be accelerated to light speed c.
\\\\
The energy to do that is
\begin{equation}
E_{pMLT}=m_{pL} \ c^2= \frac{2 \ \pi^2 \ c \ e^2 \ f_p}{10^7}
\end{equation}
\\
So this energy portion is directly proportional to the frequency.

%%%%%%%%%%%%%%%%%%%%%
%%%%%%%%%%%%%%%%%%%%%
%%%%%%%%%%%%%%%%%%%%%

\section{Hydrogen spin flip}

With the knowledge of the base rotating frequency of the Hydrogen atom in ground state the total energy of the atom at $r_1$ and the energy of the created photon, coming from infinity, can be calculated. For the regular case the magnetic moment orientation of the proton and electron is parallel, leading to the lowest possible energy.
\\
For the spin flip one assumes that the magnetic spin orientation of electron and proton are antiparallel, so that the attracting force is stroger, leading to a higher frequency, and higher total energy of the atom.
\\
The spin flip itself is now the transition from antiparellel spin to parallel spin leading to a lower total energy for the atom and creating a photon with the H-spin-flip-frequency.
\\
Doing now the same calculation as before, considering this antiparallel orientation leads to a higher total energy. The difference between the two calculations results in the energy of a photon with the spin flip frequency.
\\\\
Withe help of the derived equations is also possible to calculate the energy for a spin flip photon directly, using only the frequency. 
This serves as a test for the this new theory, that no major formula error is included.
\\\\
All calculations are based a circled paths around the kernel (S-shell).  Other trajectories are not yet considered. Also the Planck Constant is not used in the calculation.

%%%%%%%%%%%%%%%%%%%%%
%%%%%%%%%%%%%%%%%%%%%
%%%%%%%%%%%%%%%%%%%%%

\section{Summary}
This new theory combines the generation of photons with the electron transitions of an atom as long as the electrons are on a circular orbit (in the s-shell).
\\This theory allows to describe and calculate the physical parameters of an electron kernel pair if there is one electron circling around a positivle charged kernel only derived from measured  atomic spectra, or the ionisation energy of the first electron. The Planck Constant h is not used in the whole theory. From the measurable frequency of atomic spectra one can conclude back to the radius of an electron around the center of mass, and can calculate with that the kinetic energy, the angular moment  and verify the frequency. Not achievable from this is the potential energy that is contained in two opposite charged particles. First, when one calculates the resulting photon of an electron transition from infinity to the ground state, the total contained energy can be determined. 
\\
So the sum of all energy contained in a pair of oppositly charged masses (atom with one electron) can be calculated.
In the ground state of an atom the kinetic energy equals the potential energy. Also for all photons the contained potential energy has exactly the same amount as the kinetic energy.
\\\\
The total energy  $E_{\infty}$ contained in two oppositly charged elementary particles far apart from each other is the sum of the following energy portions:
\begin{enumerate}
	\item $E_{akin}$ the kinetic rotation energy of the two atom particles in ground state
	\item $E_{apot}$ the potential energy of the two particles in ground state (this equals the kinetic energy)
	\item $E_{pkin}$ the kinetic photon rotation energy (this is exactly half of the atom kinetic rotation energy)
	\item $E_{ppot}$ the potential photon energy (this equals the kinetic rotation energy of the photon)
	\item $E_{pM}$ the mass energy of the photon particle masses (acc. $E=m c^2$)
	\item $E_{pMT}$ the translational energy of the photon particle masses (they are accelerated to light speed c)
	\item $E_{pMLT}$ the translational energy of the photon Lorentz mass (this also has to be accelerated to c ). 
\end{enumerate}
It is assumed that the orientation of the magnetic moments are either parallel or antiparallel, and does not change with changing orbits.
\\\\
In the Bohr model the energy $E_{\infty}$ was set to zero(0) as reference point.
\\\\
With the derived formula set it is now possible to calculate $E_\infty $ of atoms with one electron in orbit when you know the spectral lines and the orbit numbers for the transition. Then you can calculate back to the rotation frequency of the atom in ground state. With that number you can calculate the electron rotation radius in ground state.  From that you can calculate the kinetic rotation energy of the atom that equals the potential energy. Half of that equals the kinetic rotation energy of the photon that also equals the potential energy of the photon. Furtheron you can calculate from the kinetic photon energy the photon mass energy, the translational energy of the photon particle mass. The translational energy of the Lorentz mass can be calculated from the rotation frequency of the atom directly (the Lorentz mass itself is included in the kinetic rotation energy of the photon). So also the total photon energy can be calculated. All these single components can be calculated from the rotation frequency of the atom in ground state.
That also means that for a photon the relation of energy and frequency can be determined. The expected value is the Planck-Constant h.
%%%%%%%%%%%%%%%%%%%%%
%%%%%%%%%%%%%%%%%%%%%
%%%%%%%%%%%%%%%%%%%%%

\section{Discussion, Questions, and Outlook}

%%%%%%%%%%%%%%%%%%%%%
%%%%%%%%%%%%%%%%%%%%%
%%%%%%%%%%%%%%%%%%%%%
\subsection{Mass equivalence of Energy and Mass}
In todays theory it is assumed that each energy content is equivalent to mass and is with that mass underlying gravity.
In this New Theory rotational energy that is caused by static forces (resulting from potential energy) and rotational energy caused by magnetic dipol moments are not underlying gravitation and do not create gravitation. Only the rotational energy portion that is caused by the Lorentz Force is acting as mass, underlies gravitation, and creates gravitational forces.
This does not say that there is no equivalence between mass and energy, it only says that not all energy types are underlying gravity and act as gravitation. \\

%%%%%%%%%%%%%%%%%%%%%
%%%%%%%%%%%%%%%%%%%%%
%%%%%%%%%%%%%%%%%%%%%

\subsection{Math problem: Is the energy caused by the Lorentz Force\\ a source for gravitation ?}
Now a mathematical problem can be set up.\\
Looking at two electrically neutral systems ( $\sum{q}=0$ ) each containing two oppositly charged masses rotating around their center of mass with radius r. The two systems are far apart from each other (distance = d), with $d >> r$ . It is now interesting to calculate the average Lorentz Force between the two systems. For this calculation the average has to be taken for all possible rotation orientations of the two systems and also for all phase differences of the two rotating sytsems. \\
Due to the dependence of the squared relative velocity of the Lorentz Force there might be a resulting attracting force between the two systems. 
If that can be shown, it could prove that the Lorentz Force is a reason for gravitation.
\\(A numerical calculation for this effect is complex, as the relation of elctric forces compared to gravitational forces is in a range $10^{40}$.)
\\

%%%%%%%%%%%%%%%%%%%%%
%%%%%%%%%%%%%%%%%%%%%
%%%%%%%%%%%%%%%%%%%%%

\subsection{Proposed Experiment: Do photons photons consist of charged masses ?}
Use a single slit setup with a light laser beam .\\
Use a slit that is smaller than than the wavelength of the photons.\\
Apply a magnetic field parallel to the slit behind the slit.\\
Use conductive material for the screen plane and divide this plane in half, exactly in the middle parallel to the slit.\\
Isolate the two planes electrically.\\
Measure the voltage between the plates.\\
Create a photon beam hitting the slit.\\
Measure the created voltage between the isolated planes dependent on the applied magnetic field.\\
If one changes the orientation of the magnetic field by 180 degrees, then
the voltage should change the polaritiy .\\\\
This would prove that photons consist of oppositly charged masses rotating around each other.
\\

%%%%%%%%%%%%%%%%%%%%%
%%%%%%%%%%%%%%%%%%%%%
%%%%%%%%%%%%%%%%%%%%%

\subsection{Open question: Do photons exist with the same frequency\\ but different energies ?}
It is a well known fact, that the Hydrogen atom can have different frequncies and different energies in ground state. The cause is that the orientation of the magnetic moments of electron and proton are either parallel or antiparallel.\\
Question is now is this also possible with photons, that there are photons with parallel or antiparallel magnetic moments. That would cause that photons with the same frequency can have two different energies, or the other way round, two photons with identical energies can have different frequencies.
\\

%%%%%%%%%%%%%%%%%%%%%
%%%%%%%%%%%%%%%%%%%%%
%%%%%%%%%%%%%%%%%%%%%

\subsection{Limitations of the new theory}
The whole new theory presently is based on a combination of three electrical forces.
The force from the electric potential, the force from the Lorentz force, the force from magnetic moment.
For an atom the potential enrgy-difference is calculated in the same way as in the Bohr model. But this can only be an approximation valid for large distances of the rotation radius larger than $r_g$. If the radius gets smaller, the error gets bigger, as for r=0 the potential energy would go to infinity, what cannot be.
\\\\
\textbf{This is due to the fact that the atom kernel and the electron \\ are assumed to be dimensionless point charges.}
\\\\
So in reality one can expect a further force that causes the electric force to vanishes to zero (0) at $r_g>r>0$. So also the potential energy can be expected to go to zero (0) at a radius larger than zero (0).  The same is true for the magnetic dipol force, also this math expression assumes a radius for the magnetic dipol of 0, what also cannot be. So the whole model gets inaccurate the closer it gets to small radii.
An idication might be that for the Hydrogen atom the radius is about 18.000 time larger then $r_g$, very accurate results can be expected, much better than for the Neutron, where the rotation radius is slightly below $r_g$. \\\\
New with this new theory it is possibilty to calculate the total potential energy of two opposit charged particles at infinite distance, that must equal the energy contained in the formed atom (ion) plus the total energy carried away by the generated photon. This calculation is possible with this new theory, limited by the just mentioned accuracy uncertainties.
\\\\

%%%%%%%%%%%%%%%%%%%%%
%%%%%%%%%%%%%%%%%%%%%
%%%%%%%%%%%%%%%%%%%%%
\subsection{Question: Is there a link to unstable orbits for radioactive atoms?}
If one calculates with this theory, and with the help of known (NIST) ionization energies for the first electron in orbit the radii of atoms, one sees that all radiactive elements have a radius for the first electron in ground state $r_1$ that are smaller than $r_g$.
\\
Additionally there is one other  interesting point.
\\
If one uses the general force formula for two opposit charges, one can transform the equation to a velocity equation by multipying with $r/m$. Then one has a velocity equation with the components
\begin{itemize}
	\item $v_{tot}$ for the total velocity that can be expressed as $v_{tot}^2=v_L^2+v_m^2$\\
	With this equation an immediate question apears: are the two velocities perpendicular on the rotation sphere ?
	\item $v_L$ caused by the static force in combination with the Lorentz force
	\item $v_m$ caused by the magnetic moments in combination with the Lorentz force
\end{itemize}
The relation between the different velocity components show that for all radioactive elements starting at element with atomic number 84 (PO) the relation of $v_L/v_m$ is smaller than $2 \ \pi$.\\
Question: Is this a condition for unstable orbits of the electron in ground state ?
%%%%%%%%%%%%%%%%%%%%%
%%%%%%%%%%%%%%%%%%%%%
%%%%%%%%%%%%%%%%%%%%%


\subsection{Conclusions and open questions for the structure of electrons and protons}
In this new theory electron as well as proton must be much smaller than than $r_g$ or even smaller than a neutron.
It uses as model for photons with rotating particles. The proton and electron have a magnetic moment. Are these moments also generated by rotating charged masses ? If so, the single charges must be much smaller than the elementary chage e. Only then a rotation radius smaller than $r_g$ can be achieved. What can be the mass of these rotating subatomic particles ?
How much mass is contributed by the subatomic particles itself and how much is contributed by the Lorentz energy ?
So here a complete new field of research can be opened up. What can be used or concluded from latest particle physics ?
Is this theory extendible to subatomic particles ? will that not violate the uncertainety principle ?  Is the uncertainety priciple also valid in this new theory ?  In the eyes of the author the uncertainty principle is a consequence of the theory of relativty and quantum physics. 


%%%%%%%%%%%%%%%%%%%%%
%%%%%%%%%%%%%%%%%%%%%
%%%%%%%%%%%%%%%%%%%%%


\subsection{Tasks to support this new theory}
With the Theory of Relativity (SR and GR) many observations that could not be explained with classical physics got an explanation. 
Now one can try to explain all those observations with this new theory. This is a tremendous task. It may be that this new theory might explain even other observations, that are not yet expainable with SR or GR. 
Is it possible to apply or combine this new theory with quantum physics ? 

%%%%%%%%%%%%%%%%%%%%%
%%%%%%%%%%%%%%%%%%%%%
%%%%%%%%%%%%%%%%%%%%%

\section{Hint for numerical calculation}
The new theory leads to equations of 3rd order (cubic equations). e.g. calculate the radius from the ionization energy, or from the atomic spectra. In all these cases complex 3rd order equations have to be be solved.
With todays standard calculation programs this can easily be achieved numerically. In most of the cases this is possible by using only two cells in a spread sheet that are linked via circle conditions. No additional numerical approximation program is necessary.

%%%%%%%%%%%%%%%%%%%%%
%%%%%%%%%%%%%%%%%%%%%
%%%%%%%%%%%%%%%%%%%%%

\newpage

%%%%%%%%%%%%%%%%%%%%%
%%%%%%%%%%%%%%%%%%%%%
%%%%%%%%%%%%%%%%%%%%%

\begin{thebibliography}{100}
	\bibitem [1]{Jackson}   Classical Electrodynamics,  3rd Edition
	\\
	John David Jackson
	\bibitem [2]{Heinrich Hertz}   Untersuchung Über Die Ausbreitung Der Elektrischen Kraft 1894
	\\
	Heinrich Hertz
\end{thebibliography}

\end{document}